
{\color{blue}4/12}

%I.e. for every local $f\in \cal B_0$, $\mu(R_\al f)=\mu(f)$ for all Gibbs states $\mu$.

It suffices to prove in 2D there exists $D$ such that for all Gibbs states $\mu$,
\be
\mu(R_\alpha f) \le D\mu(f)
\ee
for all positive $f\in \mathcal{B}_0$.
%rotation invariance of a priori measure
%still get a Gibbs measure.
Note that the measure $\widetilde{\mu}(f) = \mu(R_\alpha f)$ is also a Gibbs state. 

To see this, consider a finite volume Gibbs state
\be
\mu_{\Lambda}(f) = \int f(\sigma_\Lambda) \frac{e^{-\beta H^{\text{b.c.}}_\Lambda(\sigma_\Lambda)}}{Z_{\Lambda}^{\text{b.c.}}} \rho_0(d\sigma_\Lambda).
\ee
Applying a rotation to this gives
\begin{align}
\widetilde{\mu}(f) := \mu(R_\alpha f)&= \int f(\underbrace{R_\alpha\sigma_\Lambda}_{\eta}) \frac{e^{-\beta H^{\text{b.c.}}_\Lambda(\sigma_\Lambda)} }{Z_{\Lambda}^{\text{b.c.}}} \rho_0(d\sigma_\Lambda)\\
&=\int f(\eta) \frac{e^{-\beta H^{R_\alpha^{-1}(\text{b.c.})}_\Lambda (\cancel{R_{\alpha}^{-1}}\eta)} }{Z_{\Lambda}^{\text{b.c.}}} \rho_0(d\cancel{R_{\alpha}^{-1}}\eta)\\
&= \mu^{R_{\alpha}\left( \text{b.c.} \right)}(f).
\end{align}
(The boundary conditions are rotated by $-\alpha$.)
The DLR condition is invariant under rotations (exercise).
%identical or mutually singular.

{\color{red}See lemma 11.5 in notes. Change $D$ to $C$ and $<$ to $\ge$ in Lemma 11.5.}

%what distinguishes is not local function, but function measurable at $\iy$.
%Choosing a box large enough, you seldom see a difference in magnetization.

%\begin{lemma}
%For any system on $\Z^D$ with $O(2)$ symmetries of finite range interactions set
%\be
%\ab{
%\pdxy{}{}
%}
%\ee
%\end{lemma}

\begin{proof}
Let 
\be
\theta(x) = \pi \begin{cases}
1, &\left\Vert {x}\right\Vert\le L_0\\
\frac{2L-\left\Vert {x}\right\Vert_{\infty}}{L}, & L_0\le \left\Vert {x}\right\Vert\le 2L_0\\
0, &\left\Vert {x}\right\Vert_2\ge 2L_0.
\end{cases}•
\ee
(We rotate the configuration in such a way so that the rotation does not extend to $\infty$.)

%\hat R not a uniform rotation, but depends on position.
We will consider the rotation $\widehat{R}_{\frac{\alpha}{\pi}\theta}$ (we use a hat to emphasize that the rotation depends on $x$). 
In the box $\left\Vert {x}\right\Vert\le L_0$, $\widehat{R}_{\frac{\alpha}{\pi}\theta}$ acts as uniform rotation:
\begin{align}
\mu(R_\alpha f) &= \mu(f(\widehat{R}_{\frac{\alpha}{\pi}\theta}\sigma))\\
&=\int \left[ {
\int_{\Omega_{\Lambda}(2L_0)} 
f(\underbrace{\widehat{R}_{\frac{\alpha}{\pi}\theta} \sigma_{\widetilde{\Lambda}}}_{\sigma_{\widetilde{\Lambda}}'}) \frac{e^{-\beta H_{\widetilde{\Lambda}}(\sigma_{\widetilde{\Lambda}}|\sigma_{\widetilde{\Lambda}^c})}}{Z_{\widetilde{\Lambda}}(\sigma_{\widetilde{\Lambda}^c})} \rho_0(d\sigma_{\widetilde{\Lambda}})
} \right] \rho(d\sigma_{\widetilde{\Lambda}^c})\\
&=\int \left[ {
\int_{\Omega_{\Lambda}(2L_0)} 
f(\sigma_{\widetilde{\Lambda}}') \frac{e^{-\beta H_{\widetilde{\Lambda}}(\widehat{R}_{\frac{\alpha}{\pi}}\sigma_{\widetilde{\Lambda}}'|\sigma_{\widetilde{\Lambda}^c})}}{Z_{\widetilde{\Lambda}}(\sigma_{\widetilde{\Lambda}^c})} \rho_0(d\sigma_{\widetilde{\Lambda}}')
} \right] \rho(d\sigma_{\widetilde{\Lambda}^c})
\end{align}
When you carry out a gradual rotation, the energy does not change by a lot.
We have
\begin{align}
\mu(R_\alpha f) &= \mu(\widehat{R}_{\frac{\alpha}{\pi}\theta}f) \\
&= \mu(fe^{-\beta [H(\widehat{R}_{\frac{\alpha}{\pi}\theta}\sigma) - H(\sigma)]}).
\end{align}
%If we allow self to proceed by leaps of imagination, can arrive here immediately.
Formally, we rewrote
%infinite, go through finite value truncation.
%taper to 0 in a finite distance
\be
f(0) e^{-\beta H(\widehat{R}^{-1}\sigma)} \frac{\rho_0(d\sigma)}{Z}
= e^{-\beta[R^{-1}H - H]} e^{-\beta H(\sigma)}\frac{\rho_0(d\sigma)}{Z}
\ee
Note $R^{-1}H-H$ is finite; what is left is to estimte it. We estimate it in the case $H=-\sum_{|x-y|=1}\underline{\sigma}_x\cdot \underline{\sigma}_y$. We calculate
\begin{align}
H(R^{-1}\sigma) - H(\sigma) 
&= - \sum_{\text{nearest neighbor}} \left[ {
(R_{\theta(x)}^{-1}\underline{\sigma}_x)\cdot 
(R_{\theta(y)}^{-1}\underline{\sigma}_y) - \underline{\sigma}_x\cdot \underline{\sigma}_y
} \right]\\
&= \sum_{\text{nearest neighbor}} [\underline{\sigma}_x \cdot \underline{\sigma}_y - \underline{\sigma}_x \cdot R_{\theta(x) - \theta(y)} \underline{\sigma}_y]\\
&= \sum_{\text{nearest neighbor}} [\alpha(\underline{\sigma}_x\cdot \mathcal{L} \underline{\sigma}_y)]\\
&=\sum_{\text{nearest neighbor}} [\frac{\alpha}{\pi} \delta_1(H) + \delta_2(H)],
%taylor expansion 
\end{align}
where $\delta_i$ is the $i$th order correction, from the Tylor expansion. Then 
\begin{align}
\mu(R_\alpha f) &= \mu(\widehat{R}_{\frac{\alpha}{\pi}\theta}f)\\
&= \mu(f e^{-\beta [H(\widehat{R}^{-1}_{\frac{\alpha}{\pi}\theta}\sigma) - H(\sigma)]}\\
&= \mu(f e^{[\frac{\alpha}{\pi} \delta_1(H) + \delta_2(H)]})
\end{align}
%boundary increases
%softly rotate. drop $L^{D-1}$ to $L^{D-2}$
%If dimension $\le 2$ the bounded.
with $|\delta_2(H)|\le C\sum |\theta(x)-\theta(y)|^2 \le CL^{D-2}$. 

If the dimension is $D\le 2$ this is bounded. The soft rotation dropped this quantity from $L^{D-1}$ to $L^{D-2}$.

However, $\delta_1(H)$ might be bigger; a priori we only have $|\delta_1(H)|\le C\sum|\theta(x)-\theta(y)|=O(L^{D-1})$.
The idea is that we can rotate ``from the other side." To deal with the first order term,
\begin{align}
%\mu(f) &= 
\mu(f) &= \mu(f e^{-\frac{(2\pi-\alpha)}{\pi}\delta_1(H) + \widetilde{\delta}_2(H)})
\end{align}
The second-order term is controllable, the first-order term is hard to control. We have two expressions where the coefficient of the first-order terms has different signs. Let convexity work for us.

Consider 
\be
Q(t) = \mu(f e^{(1-t)[\frac{\alpha}{\pi}\delta_1(H)+\delta_2(H)]} e^{t[-\frac{(2\pi - \alpha)}{\pi}\delta_1(H) + \widetilde{\delta}_2(H)]}).
\ee
The function $\ln Q(t)$ is convex in $t$ with $Q(0)=Q(1) = \mu(R_\alpha f)$. Thus by convexity, or H\"older's inequality,
%convexity or Holder
for all $0\le t\le 1$, 
\be
Q(t) \le Q(0)^{1-t}Q(1)^{t} = \mu(Rf).
\ee
%Both terms are uniformly bounded
$(1-t)\alpha - t(2\pi - \alpha) = 0$. At $t=t_\alpha$,
\be
Q(t) = \mu(f) e^{(1-t) \delta_2+ t\widetilde{\delta}_2} \ge e^{-CL^{D-2}} \mu(f)
\ee
giving us $\mu(R_\alpha f) \ge e^{-CL^{D-2}}\mu(f)$.

%Expected vlue of rotatied is identical to original times difference produced 
%look at cost of rotating spins gently.
\end{proof}
What gave physicists the intution to look at second order term is that at low temperature, when the spins line up, the first order term vanishes. The second-order term is $\frac{L^d}{L^2}$. But there is chaos and term-by-term it's not true that the first-order term can be ignored. We sidestep this in the analysis.
%If you ask how the sysem responds to gentle rotation, the first derivative should not play a role.

If $D>2$, the bound says nothing.


The theorem implies that the pressure $\Psi(\beta, \underline{h})$ is differentiable in $\underline{h}$ at $\underline{h} = \underline{0}$. We know it's symmetric, but it could have a kink or be smooth with zero derivative. The theorem tells us it is smooth.

The argument works equally well for quantum systems and for systems with disorder. For a while people had two predictions based on 2 theories. Disordered systems in $D$ dimensions behaved as ordered systems in $D-2$ dimensions. Imry and Ma predicted that for $D\le 4$ there is no symmetry breaking. it uses this technology with bootstrapping.
%instead that continuous symmetry breaking is pushed 

Rigorous proofs arrived before physicists had a consensus.
%\begin{theorem}[Imry-Ma Theorem]
%
%\end{theorem}

Symmetry breaking occurs in dimensions $D\ge 3$. %reflection positivity.

