
{\color{blue}3-22}

We can use the transfer matrix method for 1D systems with finite range correlations (of distance $R$), not just nearest-neighbor interactions. Let the states be $\tau_x=\{(\sigma_x,\sigma_{x+1},\ldots, \sigma_{x+R})\}$. Switching to this space, we can now express the energy in terms of nearest neighbor interactions,
\be
H=\sum_n \Phi(\tau_n,\tau_{n+1}).
\ee

%aggregating some states.

\subsection{Ihara graph zeta function}

\begin{definition}
A \textbf{path} in a graph $\mathcal{G}$ is a set of edges $(x_1,x_2),\ldots, (x_n,x_{n+1})\in \mathcal{E}$. For an \textbf{oriented path}, the direction matters. In a \textbf{closed path}, $x_{n+1}=x_1$. An \textbf{oriented loop} is a closed oriented path where we don't care about the starting point, i.e., an equivalence class of paths modulo cyclic permutation. Two paths are equivalent here if $y_i=x_{i+h}$, with indices taken modulo $n$.

Edges are allowed to repeat.

For a path $p$, denote by $|p|$ the number of edges in $p$.
\end{definition}
We would like to eliminate paths which are powers of more elementary paths, complete repetitions of a primitive path. A path is primitive if it is not a power of a shorter path.

\begin{definition}
A path is \textbf{primitive} if it cannot be written as $q^k$, $k>1$, where $q$ is a closed path.
\end{definition}
This may remind you of prime numbers. Any number can be written as a product of numbers which cannot be decomposed. People applied number-theoretic ideas to graphs.
%People started thinking about number-theoretic

\begin{theorem}
Let $M$ be a matrix indexed by sites of a graph $\mathcal{G}$.
For $\left\Vert {M}\right\Vert$ small enough, %hen
\be
\det(\mathds{1}-M) = \prod_{p^{*}} [1-\chi_\mu(p)]
\ee
where the product is over equivalence classes $p^*$ of primitive oriented loops.
%matrix nonzero only for edges. don't hopscotch crazily.

Here 
\be
\chi_p=\prod_{n=1}^{|p|} M_{x_{n},x_{n+1}}.
\ee
\end{theorem}
%invar under cyclic reparametrization.

On the RHS, when we open the brackets, we can worry about convergence. For matrices of small norm, it is convergent. To define the function for all $\lambda$, note the theorem establishes that for small $\lambda$,
\be
\det(\mathds{1}-\lambda M) = \prod_{p^{*}} [1-\lambda \chi_\mu(p)].
\ee
Now we can take the analytic continuation.

Note the LHS is a finite calculation (and a multilinear function in entries of $M$), while the RHS is an infinite product/sum.

%mats with no loop
%for $\la$ small enough.

It follows from the proof that the RHS is a convergent product (for $\left\Vert {M}\right\Vert$ small enough).
\begin{proof}
First assume $\left\Vert {M}\right\Vert<1$. (This is the operator norm, $\left\Vert {M}\right\Vert=\max_v\frac{\left\Vert {Mv}\right\Vert}{\left\Vert {v}\right\Vert}$.) Then
\be
\ln \det(1-M)  = \text{tr}\ln (\mathds{1} -M).
\ee
For Hermitian matrices, prove this by diagonalization: the LHS is $\ln \prod_j (1-\lambda_j) = \sum_j \ln (1-\lambda_j)$. 

For the general case, we can still use this method. Write $M=S^{-1}TS$ where $T$ is upper triangular. If the diagonal entries are $\lambda_1,\ldots, \lambda_n$, the determinant is still $\prod_j(1-\lambda_j)$. 
%the only factor in the product is the diagonal.
%operator norm.

Expanding gives
\begin{align*}
\ln \det(\mathds{1}-M) &= -\sum_{n=1}^{\infty} \frac{1}{n} \text{tr}(M^n) \\
%&= -\sumo k{\iy} \sum_{p^*} \rc{k|p^*|} k \chi_n(p^*)^k
%elem enumeration for closed paths
%conribut at rate proportion to nmber of places where can start, total length / k.
&= \sum_{x = (x_0,\ldots, x_n)} \frac{1}{n} \left( {\prod_{j=1}^n M_{x_{j+1},x_j}} \right) \\
%&= \sumo k{\iy}\pa{\sum_{x\text{ primitive}}M_{x_{j+1},x_j}}}\\
%&= \sum_x \sumo k{\iy}
&=-\sum_{k=1}^{\infty}\sum_{p^*} \frac{1}{k|p^*|} \chi_M(p^*)^k\\
&=-\sum_{p^*} \ln (1-\chi_M(p^*)).
\end{align*}
$k$ comes from properly weighing the volume of the path.
%This is a sum over closed paths
\end{proof}

We want to avoid backtracking. %Think of $\cal L$
%\begin{theorem}
%Let $\cal E$ be the graph of oriented edges of $\cal G$.
%%backtracking and tadpoles
%%powers generate closed walks to avoid backtracking.
%
%\end{theorem}
\begin{thm*}[Theorem~\ref{thm:sherman}]
Let $\mathcal{E}$ be the graph of oriented edges of a planar graph. Let 
\begin{align*}
\mathcal{L}_{e,e'} = \mathds{1}[O(e')=t(e)]K_e e^{\frac{i\angle(e',e)}{2}}
\end{align*}
For any oriented closed path,
\begin{align*}
\prod_j \mathcal{L}_{e_j,e_{j+1}} = \prod_i K_e(-1)^{w(p)}
\end{align*}
%winding number.
with 
\be
(-1)^{w(p)} n= -(-1)^{n(p)}.
\ee
where $w(p)$ is the winding number and $n(p)$ is the number of crossings. 

\end{thm*}
%take 
%If cross sel, at that poit rerout to avoid the path
%reduces winding number of 2
We have
\begin{align*}
\prod_{p^*} [1-\chi_{\mathcal{L}}(p^*)] 
&=\sum_{\mathcal{P}= (p_1^*,\ldots, p_n^*)} \prod_j (-1)^{n(p_j)} k[p]\\
&=\sum_m\pi_e K_e^{m_e}W_m
%n?
%characterize by not thinking about the ... but the frequency of edges travesed.
%realized through collections of paths.
\end{align*}
where $W_m$ is the number of of paths with a given frequency of edges traversed. 

\begin{lemma}[Kac-Ward]
We have
\be
\sum_{P: \max_e(m(P))= 1} (-1)^{n(P)}K(P) = \sum_{\Gamma:\partial \Gamma = \phi} \prod_{e\in \Gamma} K_e = Z(\beta)
\ee
for $K_e=\tanh (\beta J_e)$. 
\end{lemma}

Each vertex has 3 possibilities.