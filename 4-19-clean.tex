
{\color{blue}4-19-16: We continue talking about continuous symmetry breaking.}

We will cover the infrared bound, reflection positivity, and the chessboard inequality.

Now the symmetry of the lattice will play a role.

Here $H=-\sum_{\{x,y\}} J_{x-y} \underline{\sigma}_x\cdot \underline{\sigma}_y$. We first cover the nearest-neighbor case. Our argument is not very robust; it only covers the nearest-neighbor case and several other special cases. It would be an achievement to find a generalization.

We pass to the Fourier transform. A basis for the space of periodic functions is 
\be
\frac{1}{\sqrt{|\Lambda|}}e^{i\underline{k}\cdot \underline{x}} ,\qquad \underline{k} = \frac{2\pi}{L} (n_1,\ldots, n_d).
\ee
We scale to take the $(0,L]^d$ grid into the grid $(-\pi,\pi]^d$ with intervals $|\Delta k_i| = \frac{2\pi}{L}$. 

%Does not mix between different momenta spaces.
Let $\widehat{\sigma}(p)$ be as in~\eqref{eq:csb-f}. The Fourier expansion of $\sigma$ is
\begin{align}
\sigma_x &= \frac{1}{\sqrt{|\Lambda|}} \sum_k^* e^{-ikx}\widehat{\sigma}(k).
%\\ H &= \sum_{k\in (-\pi,\pi]^k}^* \cal E(k)|\wh \si(k)|^2.
\end{align}
Letting $H$ be as in~\eqref{eq:csb-h1}, we find that the momentum representation of $H$ is given by~\eqref{eq:csb-h}.
Think of~\eqref{eq:csb-h1} as the discrete analogue of the integral of a gradient. Similarly,~\eqref{eq:csb-h} is the discretized version of an integral.
%Certain mnemonic devices help. 

We show that
\be
\widehat{\underline{\sigma}}(k) = 2\sum_{j=1}^{d} \sin^2\left( {\frac{k_j}{2}} \right) \approx \frac{1}{2} |\underline{k}|^2.
\ee
We calculate
\begin{align}
H &= -\frac{1}{2}\left( {\frac{1}{\sqrt{|\Lambda|}}} \right)^2\sum_{x,y} \sum_{\underline{k}, \underline{k}'} \widehat{\underline{\sigma}}(\underline{k})\widehat{\underline{\sigma}}(\underline{k}') e^{i\underline{k}\cdot \underline{x}} e^{-i\underline{k}'\cdot \underline{y}}J_{x-y}\\
&=-\frac{1}{2}\frac{1}{|\Lambda|}  
\sum_{x,y} \sum_{\underline{k}, \underline{k}'} \widehat{\underline{\sigma}}(\underline{k})\widehat{\underline{\sigma}}(\underline{k}') e^{i\underline{k}\cdot (\underline{x}-\underline{y})} e^{-i(\underline{k}-\underline{k}')\cdot \underline{y}}J_{x-y}\\
&=-\frac{1}{2}\frac{1}{|\Lambda|} \sum_{u,y} \sum_{\underline{k}, \underline{k}'} \widehat{\underline{\sigma}}(\underline{k})\widehat{\underline{\sigma}}(\underline{k}') e^{i\underline{k}\cdot \underline{u}} e^{-i(\underline{k}-\underline{k}')\cdot \underline{y}}J_{u}\\
&\quad \text{using }\sum_{\underline{k},\underline{k}'} e^{i(\underline{k}-\underline{k}')\cdot\underline{y}} = \frac{1}{|\Lambda|} \delta_{k,k'}\\
&=-\frac{1}{2}\sum_{\underline{k}} |\widehat{\sigma}(k)|^2 \widehat{J}(k)\\
\widehat{J}(k) &= \sum_u e^{i\underline{k}\cdot \underline{u}} J_{\underline{u}} \\
&=\sum_{j=1}^{d} (e^{ik_j}+e^{-ik_j})\\
&=%2
-\frac{1}{2} \sum_{j=1}^{d} \cos(k_j)\\
&= -\sum_{j=1}^{d} \left( {1-2\sin^2\left( {\frac{k_j}{2}} \right)} \right)
%CHECK FACTORS HERE.
\end{align}

%fluctuate. phases change. 
Looking at this intuitively, %when we pull out a configuration and sum over volume, 
how big can we expect to be $\sum_{x\in \Lambda} e^{ikx}\sigma_x$? The Central Limit Theorem suggest that the order of magnitude should be $\sqrt{|\Lambda|}$, making $\widehat{\sigma}(k)=O(1)$ (in high temperature). The energy $H$ is given by a sum of terms of order 1, hence is of order $O(|\Lambda|)$. 

This leads us to the right ``infrared behavior." $\mathcal{E}(\underline{k}) = -\widehat{J}(k) + \widehat{J}(0)$. (?)

Plancharel's identity tells us
\begin{align}
\frac{1}{|\Lambda|}\sum_x |\sigma_x|^2 &= 
\frac{1}{|\Lambda|}\sum_{\underline{k}}^* |\widehat{\sigma}(k)|^2\\
1 &= \frac{1}{|\Lambda|}|\widehat{\sigma}(0)|^2 + \frac{1}{|\Lambda|} \sum_{\underline{k} = \frac{2\pi}{L} \underline{n}\in (-\pi, \pi]^d}|\widehat{\sigma}(k)|^2
\end{align}
Think of this as a Riemann sum approximation to the integral $\frac{1}{(2\pi)^d}\int_{(-\pi,\pi]^d}\,du$. (Each cube has volume $\left( {\frac{2\pi}{L}} \right)^d$.
%Volume $\pf{2\pi}{L}^d$.

\subsection{Infrared bound}
\begin{theorem}[Infrared bound]
For the nearest neighbor (and other reflection positive models), \be\left\langle {|\widehat{\underline{\sigma}}(k)|^2}\right\rangle_{\beta, L}^{\text{p.b.c}}\le \frac{1}{2\beta \mathcal{E}(k)}.\ee
Equivalently,
\begin{equation}\label{eq:csb-fs}
\mathcal{E}(k) \left\langle {|\widehat{\underline{\sigma}}(k)|^2}\right\rangle \le \frac{1}{2\beta}.
\end{equation}
\end{theorem}
Near 0, $\mathcal{E}(k) \approx \frac{1}{2}| k|^2$. Thus the infrared bound diverges at 0.

This is a very physical statement. %it's meaningful in physical terms. 
The physical interpretation is that $\mathcal{E}(k) \left\langle {|\widehat{\underline{\sigma}}(k)|^2}\right\rangle$ is the mean energy in the $k$th mode. %The equipartition law says you have $\rc 2 kp$ per mode. 

Take the mean value of~\eqref{eq:csb-fs} to obtain
\be
1 = \left\langle {\frac{1}{|\Lambda|} |\widehat{\sigma}(0)|^2}\right\rangle + \frac{1}{|\Lambda|} \sum_{\underline{k}=\frac{2\pi}{L} \underline{n}\in (-\pi, \pi]^d} \underbrace{\left\langle {|\widehat{\sigma}(k)|^2}\right\rangle}_{\frac{1}{2\beta} \frac{1}{\mathcal{E}(k)}}.
\ee
%(Small values of $k$ is the infrared regime.)

If $d>2$, although the bound diverges at 0, it is integrable. 

We have
\begin{align}
1&\le \frac{1}{|\Lambda|} \left\langle {|\widehat{\sigma}(0)|^2}\right\rangle +\left[ { \int \frac{1}{(2\pi)^d} \text{``} \int_{(-\pi,\pi), k\ne 0} \text{"}\frac{1}{\mathcal{E}(k)} } \right]\frac{1}{2\beta}\\
\lim_{L\to \infty} \frac{1}{L^d} \left\langle {|\widehat{\sigma}(0)|^2}\right\rangle_{L,\beta} 
&\ge 1-\frac{1}{2\beta} \underbrace{\frac{1}{(2\pi)^d} \int_{(-\pi, \pi]^d} \,du^d \frac{1}{\mathcal{E}(k)}}_{=:C_d<\infty \text{ when }d>2}
%if $\be$ small, no news.
%if $\be$ is large
%0 momentum requires positive mass.
%bose-einstein condensation. mass weight to particular mode.
\end{align}
This says that ``zero momentum requires positive mass." Compare this to Bose-Einstein condensation, when there is a point mass at a particular mode.

%1/\sqrt{|\La|}, add another
This is more informative if $\beta$ is large. For $\beta >2 C_D$, 
\be
\left\langle {\left| {\frac{1}{|\Lambda|}\sum \sigma_x} \right|^2}\right\rangle=
\frac{1}{|\Lambda|}\left\langle {|\widehat{\sigma}(0)|^2}\right\rangle_{L,\beta} \ge \left( {1-\frac{C_D}{2\beta}} \right)>0.
\ee
Also, 
\begin{align}
\left\langle {\underline{\sigma}_0\underline{\sigma}_x}\right\rangle &= 
\left\langle {\left( { \frac{1}{\sqrt{|\Lambda|}} \sum_{k'}^* \widehat{\sigma}(k')} \right)
\left( {\frac{1}{\sqrt{|\Lambda|}} \sum_k^* e^{-i\underline{k} \cdot x} \widehat{\sigma}(k)} \right)}\right\rangle\\
\left\langle {\underline{\sigma}_0\underline{\sigma}_x}\right\rangle&= \frac{1}{|\Lambda|} \sum_k\left\langle { |\widehat{\sigma}(k)|^2}\right\rangle e^{-i\underline{k}\cdot \underline{x}}\\
%get info by apply Plancharel
%\an{\rc{\sqrt{|\La|}}\sum
\frac{1}{|\Lambda|}\sum_x\left| {\left\langle {\underline{\sigma}_0\underline{\sigma}_x}\right\rangle } \right|^2 &= \frac{1}{|\Lambda|^2} \left\langle {\left| {\widehat{\sigma}(0)} \right|}\right\rangle^2
+ \text{``}\frac{1}{2\pi} \int_{(-\pi, \pi]^d} \,du\text{"} \left\langle {|\widehat{\sigma}(k)|^2}\right\rangle e^{ikx}.
%\rc{|\La|} \sum_{x\in \La} |\an{\si_0,\si_x}|^2 
%&=
\end{align}
The function has an integrable singularity at the origin. A Riemann-Lebesgue argument shows this tends to 0 as $x\to \infty$ with $|x|\le \frac{L}{2}$. (I.e., take $\lim_{|x|\to \infty}\lim_{L\to \infty}$.)
%for each $x$ take $L\to \iy$. Then $|x|\to \iy$.
%1\ll 
%For large volumes, this quantity tends to 0. The 0-momentum mode has massive mean value. 

We get that 
\be\left\langle {\underline{\sigma}_0,\underline{\sigma}_x}\right\rangle_{L,\beta} =\Omega(1)
\ee for all $L$ (long-range order).

To saturate the sum rule, we have to get massive density/volume-type occupation at 0.

The IR bound gives
\begin{align}
Z&=\int e^{\frac{\beta}{ }2 \sum_{x,y} J_{x-y} |\underline{\sigma}_x - \underline{\sigma}_y|^2 }\,\rho_0(d\sigma)\\
%\rc{2^{|\La|}} \sum_{\si:\si_x=\pm1} \cdots \equiv \int \cdots \rh_0\,d\si
%yank different spins out of $\pm1$ by shifting. What is the optimal shift? The best you can do is do nothing.
Z(\underline{h}) &= \int e^{\frac{\beta}{2}\sum_{x,y} J_{x,y}|(\underline{\sigma}_x + \underline{h}_x) - (\underline{\sigma}_y + \underline{h}_y)|^2}\, \rho_0(d\sigma).
\end{align}
We yank different spins out of $\pm1$ by shifting. What is the optimal shift? The best you can do is do nothing.

We claim $Z(\underline{h}) \le Z(\underline{0})$.

Then $\frac{\partial^2 }{\partial {x}^2} Z(\lambda \underline{h})|_{\lambda=0} \le 0$, giving the IR bound.

\subsection{Reflection positivity}
Recall the Cauchy-Schwarz inequality: If $\left\langle {\cdot, \cdot}\right\rangle$ is a bilinear positive form ($\left\langle {\psi,\psi}\right\rangle\ge 0$), then
\be
|\left\langle {\psi,\varphi}\right\rangle|\le |\left\langle {\psi|\psi}\right\rangle|^{\frac{1}{2}}|\left\langle {\varphi,\varphi}\right\rangle|^{\frac{1}{2}}.
\ee
%Now we put this on steroids.

\begin{definition}
A system is \textbf{reflection positive} with respect to a plane if letting $R$ be the reflection around the plane, letting $L$ and $R$ be the left and right side of the plane, for every $f\in \mathcal{B}_L$,
\be
(Rf)(0) = f(R\sigma)
\ee
and
\be
\left\langle {\overline{F}(\sigma)  F(R\sigma)}\right\rangle\ge 0.
\ee
\end{definition}
The Schwarz inequality and reflection positivity say that for all $F,G\in \mathcal{B}_L$, 
\be
|\left\langle {F(\sigma)G(R\sigma)}\right\rangle| \le \left\langle {\overline{F}RF}\right\rangle^{\frac{1}{2}} \cdots \left\langle {\overline{G}RG}\right\rangle^{\frac{1}{2}}.
\ee

%bootstrap and give upper bound. 
The idea is to take reflections over many different planes. Consider a product over boxes $\alpha$,
\be
\left\langle {\pi_\alpha F_{\alpha}(\sigma)}\right\rangle \le \prod_\alpha\left\langle {\prod_{\alpha'} F_\alpha^{\#}(\sigma_{\alpha'})}\right\rangle^{\frac{1}{|\Lambda|}}\ge 0
\ee
%take multiple images
%chessboard.
%Taking a single function and duplicating over the system
where $F_\alpha^{\#}$ is $F_\alpha$ extended to all squares by reflection, the ``chessboard" extension. %We get the partition function of a function modified extensively.
