\def\filepath{templates}
%\def\filepath{C:/Users/holden-lee/Dropbox/Math/templates}

\input{\filepath/packages_book.tex}
\input{\filepath/theorems_with_boxes.tex}
\input{\filepath/macros.tex}
\input{\filepath/formatting.tex}
\input{\filepath/other.tex}
\input{\filepath/theorem_num.tex}

\def\name{Mathematical Physics}


\pagestyle{fancy}
%\addtolength{\headwidth}{\marginparsep} %these change header-rule width
%\addtolength{\headwidth}{\marginparwidth}
\lhead{PHY521/MAT597}
\chead{} 
\rhead{Mathematical Physics} 
\lfoot{} 
\cfoot{\thepage} 
\rfoot{} % !! Remember to change the problem set number
\renewcommand{\headrulewidth}{.3pt} 
%\renewcommand{\footrulewidth}{.3pt}
\setlength\voffset{0in}
%\setlength\textheight{648pt}

\begin{document}
%\input{\filepath/titlepage.tex}
%\maketitle
\title{Mathematical Physics}
\author{taught by Michael Aizenman}
\maketitle
%\tableofcontents

\startcontents
\printcontents{ }{-1}{}

\chapter*{Introduction}
Notes from Michael Aizenman's class ``Mathematical Physics" at Princeton in Spring 2016. 

This class is of interest to both physicists and mathematicians. Several recent Fields medals are for work related to these topics.

%don't assume knowledge
I plan to cover the following topics. The focus is on \vocab{Topics in Mathematical Statistic Mechanics}.
\begin{enumerate}
\item
\textbf{The statistical mechanic perspective}: systems can be described at the microscopic level with many degrees of freedom. We observe their collective behavior and find emergent behavior.
\item
\textbf{Thermodynamics principles}:
%fasci
%macroscopic description of physical system
In departure from mechanics, which cares about equalities like $F=ma, E=mc^2$, a unique thing about thermodynamics is that its key principle is in inequality: entropy increases.
\[
\De S\ge 0
\]
%Entropy comes up as a emergent phenomenon from the statistical mechanics description of reality.
\item
\vocab{The emergence of thermodynamics from statistical mechanics via the equidistribution assumption and the ``large deviation theory"}. %vadhan - abel prize for large deviation theory.
Mathematicians formalized the theory but the concepts were introduced earlier by physicists.
\item 
\vocab{Phase transitions}: %Consider $H_2O$. You can control it by temperature and pressure. 
A small change in parameters produces a discontinuous change in the system.
For example, a small change in the temperature of water can change it from liquid to solid. 
\item 
\vocab{Critical phenomena, critical exponents, universality classes}. Systems are macroscopically different, but the singularities you observe are given by the same power laws.
%Wilson.
\item
\vocab{Exact solution of the 2-D Ising model}. %From the perspective of physics. 
Mathematically, 3 is the hardest dimension to comprehend.
\begin{enumerate}
\item
1-D is solvable: correlations can be described by Markov chains, and can be computed. 
\item
In 2-D, the conformal group is very important. It gives many constraints on critical behavior, leading to a rich behavior. 
\item
In 3-D, this does not apply except for ongoing work finding consequences in 3-D from results in 2-D. 
\item
Anything with $\ge 4$ dimensions gets simpler. High dimensions are characterized by the fact that loop effects do not play a large role. In 1-D a simple random walk is recurrent. In sufficient high dimensions, a simple random walk is not recurrent.
The infinite-dimensional case reduces to models on trees; high dimensions exhibit the same behavior as infinite dimensions.
\end{enumerate}
In coding theory and discrete math people study phase transitions of different graphs and are interested in the same topics.

I will study the 2D Ising model through ``graph zeta functions." 
%There are analogues 
There is a relationship between them and the solution of the graph Ising model.
\item
\vocab{Stochastic geometry behind correlation functions at criticality}: in the Ising model we have a collection of spin variables $\si_x=\pm1$ for $x\in \Z^d$. The energy favors correlations; agreement among neighbors is encouraged. These correlations spread through the system. Represent this system of correlated spins with a shadow system where we play the following game: the collection of spins is decomposed at random into connected clusters. For each such decomposition, the spins are the same in each connected cluster.

Ex. students form cliques, and each clique votes independently and unanimously. Someone who doesn't see the cliques and just the votes sees clusters. The pattern becomes transparent once you know the cliques.

The cliques become larger and more fractal. There is an interesting fractal geometry which tells us about correlation functions.
\item \vocab{Scaling limits}: when we have a statistic mechanical system which macroscopically is described by a myriad of variables. We want to know properties of this substance. There is a technique for find these properties. %from the macroscopic description. 
This gives a link between statistical mechanics and field theory.
\item \vocab{Related results for quantum spin systems}.
\end{enumerate}
%random operators
%weekly packages

There is a broad spectrum of references, none of which I'll follow exclusively. 
\begin{itemize}
\item
Sacha Friedli and Yvan Velenik's online book project (on mathematical statistical mechanics) at \url{http://www.unige.ch/math/folks/velenik/smbook/index.html}. I recommend this for people totally new to the subject.
\item
David Ruelle formulated models and basic results mathematically (late 70's). His book helped physicists organize their thoughts. It became outdated quickly, but remains a good starting point and reference for the formalism.
\end{itemize}



%\setcounter{chapter}{-1}

\chapter{Introduction to statistical mechanics}

\blu{2-2-16}

\section{The equivalence principle}
\subsection{Configurations and ensembles}
One way to start is with the axioms of statistical mechanics. Instead I'll take a simple problem, see how it works, and present results in that context. There are simple problems that teach us a lot. The simplest is a lattice gas.

A lattice gas is a substrate where at each lattice site there may or may not be a particle. 

The \ivocab{configuration} is a function $n:\Z^d\to \{0,1\}$:
\[
n_x=\begin{cases}
1,&\text{$x$ is occupied},\\
0,&\text{$x$ is vacant}.
\end{cases}
\]
I'll use $L$ to denote the size of the box we are considering, and $\La\sub \Z^d$ to be a region (subset of the system). The \vocab{configuration space} is the space of possible $n$'s, $\{0,1\}^{\La}$.

We assume conversation in the number of particles, and that particles cannot overlap. %Suppose that every configuration gets equal weight. 
We make the \ivocab{equidistribution assumption}: every configuration with the same number of particles has equal probability. This gives rise to the \ivocab{microcanonical ensemble}. An ensemble is a probability measure with respect to which you do averages. We have
\[
\Pj(n_{\Om})= \fc{\one[\sum_{x\in \Om}n_x=N]}{Z}
\]
where $Z$ is a normalization constant.
Here, $\one[\text{cond}] := \begin{cases}
1,&\text{condition satisfied}\\
0,&\text{elsewhere}
\end{cases}$.

For every function $f:\Om\to \R$ assigning a real number to each configuration, define the \ivocab{microcanonical ensemble average} by
\[
\an{f}_{N,n}^{\text{Can}} = \fc{\sum_{n\in \Om} \one[n_x=N] f(n)}{\sum_{n\in \Om} \one[\sum n_x=N]}.
\]

Loosely, the \ivocab{equivalence principle} says that for any ``local function", the microcanonical average is approximately the grand canonical ensemble average
\[
\an{f}_{N,\La}^{\text{Can}}\approx \an{f}_{\mu, \La}^{\text{Gr.C}}
\] 
when we take $\mu = \fc{N}{|\La|}$.
%at suitable $\mu=\mu\pf{N}{n}$.

The \ivocab{grand canonical ensemble average} is defined as
\[
\an{f}_{\mu, \La}^{\text{Gr.C}} = \fc{\sum_{n\in \Om} e^{-\mu  \sum_{x\in \La}n_x}f(n)}{\sum_{n\in \Om}e^{-\mu \sum_{x\in \La} n_x}}
\]
(Later on we will omit superscripts where it is clear.)

\subsection{Equivalence principle: first proof}

Consider functions which depend only on a system $\La\subeq \wt{\La}$ of much smaller volume, $|\La|\ll |\wt{\La}|$. This is the sense in which the averages match up.

%sociological interest. 
The micro-canonical ensemble is draconian: the number of particles is prescribed, all other configurations get weight 0. %if don't fit bill get weight 0. 
In the grand canonical ensemble, each configuration contributes. There is a value of $\mu$ where the density is the same; at that value the local average of the draconian system is asymptotically the same at that of the more relaxed system. This $\approx$ becomes $=$ when you take the thermodynamic limit,
\bal
\wt\La & \to \Z^d\\
N & \to \iy\\
\fc{N}{\wt \La} & \to \rh.
\end{align*}

%If all that matters it the number of particles in $\La$, then we care about the induced distribution on $\La$. %What is the probability distribution 
What is the induced distribution of the micro-canonical ensemble on $\La$? Under the micro-canonical ensemble what is the probability that $n_{\La}$ (the restriction to $\La$) 
%n_{\Ga_n}
takes a particular value with $\sum_{n}n_x=k$? %If you specify a configuration in $\L$
We count the number of ways to complete the configuration in $\La^c=\wt\La\bs \La$:
\bal
\fc{|\set{n_{\La^c}}{\sum_{x\in \La^c}n_x=N-k}|}{C}
\end{align*}
where $C$ is a normalization constant.

The number of configurations of $M$ particles in volume $V$ is $\binom{V}{M} = \fc{V!}{M!(V-M)!}$. Using Stirling's approximation
\[
\ln (M!) = M(\ln M-1)(1+o(1)),
\]
%gymnastics of elementary type.
%Boltzmann grade in Vienna
%entropy
letting the \ivocab{entropy} $S$ be the logarithm of the number of configurations, %$S=k\ln W$,
\[
\binom{V}{M} = \fc{V!}{M!(V-M)!} 
=: e^{S(M,V)}
\approx e^{Vs(\rh)}
\]
(do this calculation as an exercise)
where 
\[
s(\rh) = -[\rh \ln \rh + (1-\rh)\ln (1-\rh)].
\]
Shannon also found such a formula for entropy.

This attains maximum at $\ln 2$ at $\rc2$ where it has quadratic behavior.

The implication is that if you slightly change the density, the number of configurations changes drastically. In physical substances $V$ may be $10^{23}$. The change is $e^{10^{23}\De s}$. In any average over configurations, only those at the peak contribute, ``winner takes all."

What is the probability of observing $k$ particles in the small box given $n$ in the big box?
\bal
\Pj(n_{\La}) &\approx \fc{e^{|\La^c|s\pf{N-k}{|\wt \La| - |\La|}}}{C}\\ %\quad \rh=\fc{N}{|\wt{\La}|}
%volume of completent times density.
\fc{N-k}{|\wt\La|-|\La|}&=\rh - \fc{k}{|\wt\La|-|\La|}\\
s\pf{N-k}{|\wt\La|-|\La|} & \approx s(\rh) - s'(\rh)\fc{k}{|\La^c|}
%particles here is small fraction of number of outside.
\end{align*}
Changing $k$ by a little bit affects how many particles are outside but not so much the density outside: the correction term is small. Hence for $\sum_{x\in \La}n_x$,
\[
\Pj(n_\La) = \fc{e^{|\La^c|s(\rh)}e^{s'(\rh)}k}{C}.
\]
$e^{|\La^c|s(\rh)}$ is a huge factor but it does not vary with $k$ so we can omit it. This is then ($\mu = e^{-s'(\rh)}$)
\[
=\fc{e^{-\mu k}}{\sum_{n'\in \Om_{\La}}e^{-\mu\sum_{x\in \La} n'_x}}.
\]
The rest of the system acts on the small system as a ``particle (heat) bath."

Here we used very explicit machinery.

 Note we can also apply this method where there are more energy constraints. Make a list of energy constraints; there is a generalization of the equivalence where we averaging over configurations where the constraints have prescribed values. %meet the constraint. All of them get equal weight.
Functions which depend on a small region, can be computed with Gibbs factors, $e^{-\mu N(n) - \be\cal E(n)}$ where $N(n)$ is the number of particles, and $-\mu N(n)$ is the Gibbs factor, and $\cal E(n)$ the energy.

How can we construct an alternative method without the Stirling formula?

Define
\[
Z_{\wt \La} = \sum_{n\in \Om} e^{-\mu N(n)}.
\]
This can be easily computed without Stirling. From the value of this, you can learn the value of the entropy function:
\bal
&=\sum_{n\in \Om} \prod_{x\in \wt\La} e^{-\mu \one[n_x=1]}\\
&=\prod{x\in \wt{\La}}(1+e^{-\mu}) = (1+e^{-\mu})^{|\wt \La|}.
\end{align*}
However, 
\bal
Z_{\wt \La} &=\sum_{n\in \Om} e^{-\mu N(n)}\\
&= \sum_{K\in \N} e^{-\mu K}e^{V s\pf{K}{v}}\\
&= \sum_{K\in \N} e^{V(s\pf{K}{V}-\mu \fc{K}V)}.
\end{align*}
For each $k$ count how many configurations have that value of $k$. Here $S(V,K)=Vs\pf{K}{V}$.
Suppose we find it acceptable to say the system \emph{has} an entropy.

The maximal value of $\fc KV$ takes it all: this is
\[
=e^{V\max_{\rh\in [0,1]}[s(\rh)-\mu\rh]}.
\]
This is the \ivocab{Legendre transform} of the entropy.
Let
\beq{eq:s-leg}
s^*(\mu):= \max_\rh [s(\rh) - \mu\rh] = \ln (1+e^{-\mu})
\eeq
Next time we'll discuss how to derive from this expression the formula for $s(\rh)$, using the inverse Legendre transform.

``The elementary problems are the most precious, once you absorb them they are part of your makeup."

\blu{2-4-16}

%Lattice gas
%$\La, M$ particles
%$n:\La\to \{0,1\}$
%Configuration space $\Om=\bit^{\La}$.
%Study particles within region. The fact that the total number of particles is fixed is less visible in a small region.
%For the grand canonical ensemble, we have independence.
%Can: $\Pj(n) = \fc{\one[N(n)=M]}{C}$, 
%The grand canonical ensemble $\fc{e^{-\mu N(n)}}{Z_\La(\mu)}$.
%we can compute using the entropy.
%how without stirling.
\subsection{Second proof}

The \ivocab{partition function} for the grand canonical ensemble is
\bal
Z_\La & = \sum_{n\in \Om} e^{-\mu N(n)} = (1+e^{-\mu})^{|\La|}\\
\rc{|\La|} \ln Z_{\La} &= 1+e^{-\mu}
\end{align*}
At each site there are 2 possibilities, and the different sites are independent.
We find\footnote{Mathematicians are paranoid, so we use $\sup$ instead of $\max$. For many practical purposes they are the same.}
\begin{align}%\llabel
\lim_{|\La|\to \iy} \rc{|\La|}\ln Z_{\La} &= \sup_{0\le \rh\le 1} [s(\rh) - \rh \mu].
\end{align}
%We know this is true because 
In slow motion, note that the only thing that matters in the summand is the number of particles in $n$, so let's group the summands by this. Letting $\rh=\fc{N}{|\La|}$,
\bal
Z_N&=\sum_{n\in \Om} e^{-\mu N(n)}\\
& =\sum_{n\in \Om} e^{-\mu N(n)} e^{|\La|s(\rh)}\\
& =\sum_{\rh} e^{|\La|[s(\rh)-\mu \rh]}.
%concave modified by a linear function.
%for each $\mu$ there is some maximum.
%when you look at contribution of points $\ep$ away, it changes by a factor of the volume. 2
%all terms which are $\ep$ away contribute a negligible amount.
\end{align*}
The multiplicity is the exponential of the volume times the entropy.
%Should we should multiply by the range . 
The error we are making by focusing on the maximal point rather than counting with multiplicity the near-maximal value. We lose at most a factor equal to the volume. But $\rc{|\La|}\ln |\La|\to 0$. %excessively meticulous.

We find 
%taking logs.
\[
\max_{\rh}[s(\rh) - \mu \rh] \le \rc{|\La|}\ln Z_{\La} \le \max_\rh [s(\rh)-\mu \rh] + \ub{\rc{|\La|}\ln |\La|}{\to 0}
\]
%from the partition function we can extract the entropy.
%consistent with Stirling formula? Can we do without? Yes.

Hence 
\[
\sum_{0\le \rh\le 1}[s(\rh) - \mu\rh] = 1+e^{-\mu}
\]
%This is essentially Legendre transform. It is in particular invertible.
%Stirling's approximation gave us the answer for the entropy. Any other model: you will not have the luxury of the Stirling approx argument. 
For $s(\rh) = -[\rh\ln \rh +(1-\rh) \ln (1-\rh)]$, we find the critical point by setting the derivative to 0. %\footnote{don't differentiate in public} 
Trick: $[x(\ln x-1)]'=\ln x$. We can subtract 1 from each of the logs changing the expression by a constant. Thus
\[
s'(\rh) = -\ln \rh +\ln (1-\rh) - \mu.
\]
Solving gives $\fc{\rh}{1-\rh}=e^{-\mu}$. 

\subsection{Legendre transform}
In general, a micro-canonical ensemble specifies all the conserved quantities: Particles, energies, and whatever else is there. Th grand canonical ensemble also generalizes by changing the factor to $e^{-\mu N} e^{\be H}$.
Such factors are referred to as \ivocab{Gibbs factors} %states
or Gibbs measures.
%variational characterization of those.
Next topics.
\begin{itemize}
\item
Legendre transform
\item
Convexity/concavity
\item A variational characterization of Gibbs states
\item First order phase transitions (thermodynamics and statistical mechanics perspectives)
\end{itemize}

\begin{df}
A function on $\R^k$ is \ivocab{concave} (\ivocab{convex}) if for any $x_0,x_1\in \R^k$, $0\le \la \le 1$,
\[
F(\la x_1+(1-\la)x_0)\lge \la F(x_1)+(1-\la) F(x_0).
\]
\end{df}
Concavity (convexity) means if you draw a chord between two points, it will lie below (above) the curve.
%A concave curve frowns.

%useful for variational principle
For a strictly concave function, a maximum, whenever it exists, is unique.
%Feynman: getting the sign right is the hardest thing.

\begin{thm}
For any concave function on $\R$, 
\begin{enumerate}
\item
The directional derivatives $F'(x\pm0)$ exist at all $x\in \R$. The \ivocab{directional derivative} is defined as
\bal
F'(x+0)&=\lim_{\ep\to 0^+} \fc{F(x+\ep)-F(x)}{\ep}\\
F'(x-0)&=\lim_{\ep\to 0^-} \fc{F(x+\ep)-F(x)}{\ep}.
\end{align*}
\item $F'(x-0)\ge F'(x+0)$ and $F'(x\pm 0)$ are decreasing.
\item For all but countably many values $x\in \R$, $F'(x-0)=F'(x+0)$, i.e., $F$'s different at $x$.
\item Let $F_n$ be a sequence of concave functions which converge pointwise: for all $x$, $\lim_{N\to \iy} F_N(x)=:\wt F(x)$ exists. 
Then
\begin{enumerate}
\item
 $\wt F$ is concave.
\item
At points of differentiability of $\wt F$, the derivatives also converge,
\[
F'(x\pm 0) \to \wt F'(x).
\]
\footnote{Finite energy functions are always smooth, but %as a baby's face
their limit can have discontinuous derivative.
%Converge pointwise without derivative converging.
}
\end{enumerate}
\end{enumerate}
\end{thm}

\ig{images/2-3}{.25}

Much of this generalizes to directional derivatives in $n$ dimensions.
\begin{proof}
Concavity implies that the slope of the line between $x,x+\ep$ is increasing as $\ep\to 0^+$.

\ig{images/2-1}{.25}

%The slope of the line between $x,x-\ep$ is decreasing as $\ep\to 0^-$.
To prove 3, think of the graph of the derivative at the right. The sum of any uncountable number of steps is infinite.\footnote{More rigorously, for every nonzero interval $[F'(x+0),F'(x-0)]$, we can associate with it a rational number. $\Q$ is countable}
\footnote{The set of discontinuities can be dense, ex. at all the rational numbers. In physics it was thought that this can't occur, but there are materials whose free energy discontinuities is dense in certain areas.}

\ig{images/2-2}{.25}
\end{proof}

\begin{df}
The \ivocab{Legendre transform} of a function is defined as
\[
(TG)(y) = \inf_x [y\cdot x - G(x)] = - \sup_x [G(x) - y\cdot x]
\]
\end{df}
\begin{thm}
For any function $G$, $TG$ is concave.
\end{thm}
%You can futz with $\ep$, but  from a certain angle this is immediate.
\begin{proof}
An efficient way to think about this: for each value of the parameter $x$, as a function of $y$ this is a linear function. For each $x$ we get a linear function. Define the transform by taking the infimum over that. 

Take 2 points and draw the chord between them. For each linear function the chord lies below it. 

\ig{images/2-4}{.25}
\end{proof}
I.e., we used
\begin{pr}
Let $\cal F$ be a collection of concave (e.g. linear) functions. Then $\inf_{f\in \cal F}f(x)$ is concave.
\end{pr}
\begin{df}
The \ivocab{concave hull} $\wt G$ of $G$ is defined as the smallest concave function that is at least the function value at every point:
\[
\wt G(x)=\inf \set{F(x)}{F\text{ concave, }\forall u, \,F(u)\ge G(u)}.
\]
%look at all functions that are concave and dominate it.
\end{df}

\ig{images/2-5}{.25}

\begin{thm}
For \emph{concave} $G$,
\[
T(TG)=G.
\]
In general, $T(TG)$ is the concave hull of $G$. 
\end{thm}
\begin{proof}[Proof for $G$ differentiable]
Use the fact that if $G$ is differentiable, then
\[
\inf_x [y\cdot x - G(x)]
\]
occurs at $y=G'(x)$. %learn about function in discontinuous fashion. exactly what happens in 1st order phase transitions.
\end{proof}

Note that we plotted the function and the dual function on the same graph. However, they have inverse units, for example, energy and inverse temperature. You get a lot of insights into physics if you keep track of the units.
%energy and inverse temperature. You get a lot of insights into physics if you keep track of the units. Note $x,y$ have inverse units.


Recall $\rh=\fc{N}{|\La|}$. %prob within configurations of observing something.
If all $2^{|\La|}$ configurations are given equal weight, the typical value or $\rh$, the particles per unit volume is $\rc2$. The Law of Large Numbers says that with probability 1 the ratio tends to $\rc2$. Is it possible that the density is $\rc3$? Yes, but the probability of such a density is given by the entropy: it's exponentially small, $e^{|\La|[s\prc{3} -\ln 2]}$. Anything other than $\rc2$ is a large deviation; they occur with exponentially small probability. We want to quantify the probability of large deviation events. Here is a language that people found useful.
\begin{df}
A sequence of probability measures on $\R$ is said to satisfy a \ivocab{large deviation principle} with speed $\{a_N\}$, and rate function $I$ if for each $x\in \R$, $\ep>0$,
%raghu vadhan
\[
-\inf_{|u-x|<\ep} I(u)\le 
\liminf_{N\to \iy}\rc{a_N} \ln \Pj_N((x-\ep,x+\ep]) \le \limsup_{N\to \iy} \rc{a_N} \ln \Pj_N((x-\ep,x+\ep]) \le 
-\inf_{|x-u|\le \ep} I(u)
\]
%much more complicated situations
\end{df}
For us, $I(x)=\ln 2 - s(x)$.
%This definition generalizes the situation where the previous applies.
%INSERT_HERE

%\appendix
%\input{distribution_chapters/a.tex}

%\bibliographystyle{plain}
%\bibliography{\filepath/refs}

\printnomenclature
\printindex
\end{document}