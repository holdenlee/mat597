\chapter{Introduction to statistical mechanics}

\blu{2-2-16}

\section{The equivalence principle}
\subsection{Configurations and ensembles}
One way to start is with the axioms of statistical mechanics. Instead I'll take a simple problem, see how it works, and present results in that context. There are simple problems that teach us a lot. The simplest is a lattice gas.

A lattice gas is a substrate where at each lattice site there may or may not be a particle. 

The \ivocab{configuration} is a function $n:\Z^d\to \{0,1\}$:
\[
n_x=\begin{cases}
1,&\text{$x$ is occupied},\\
0,&\text{$x$ is vacant}.
\end{cases}
\]
I'll use $L$ to denote the size of the box we are considering, and $\La\sub \Z^d$ to be a region (subset of the system). The \vocab{configuration space} is the space of possible $n$'s, $\{0,1\}^{\La}$.

We assume conversation in the number of particles, and that particles cannot overlap. %Suppose that every configuration gets equal weight. 
We make the \ivocab{equidistribution assumption}: every configuration with the same number of particles has equal probability. This gives rise to the \ivocab{microcanonical ensemble}. An ensemble is a probability measure with respect to which you do averages. We have
\[
\Pj(n_{\Om})= \fc{\one[\sum_{x\in \Om}n_x=N]}{Z}
\]
where $Z$ is a normalization constant.
Here, $\one[\text{cond}] := \begin{cases}
1,&\text{condition satisfied}\\
0,&\text{elsewhere}
\end{cases}$.

For every function $f:\Om\to \R$ assigning a real number to each configuration, define the \ivocab{microcanonical ensemble average} by
\[
\an{f}_{N,n}^{\text{Can}} = \fc{\sum_{n\in \Om} \one[n_x=N] f(n)}{\sum_{n\in \Om} \one[\sum n_x=N]}.
\]

Loosely, the \ivocab{equivalence principle} says that for any ``local function", the microcanonical average is approximately the grand canonical ensemble average
\[
\an{f}_{N,\La}^{\text{Can}}\approx \an{f}_{\mu, \La}^{\text{Gr.C}}
\] 
when we take $\mu = \fc{N}{|\La|}$.
%at suitable $\mu=\mu\pf{N}{n}$.

The \ivocab{grand canonical ensemble average} is defined as
\[
\an{f}_{\mu, \La}^{\text{Gr.C}} = \fc{\sum_{n\in \Om} e^{-\mu  \sum_{x\in \La}n_x}f(n)}{\sum_{n\in \Om}e^{-\mu \sum_{x\in \La} n_x}}
\]
(Later on we will omit superscripts where it is clear.)

\subsection{Equivalence principle: first proof}

Consider functions which depend only on a system $\La\subeq \wt{\La}$ of much smaller volume, $|\La|\ll |\wt{\La}|$. This is the sense in which the averages match up.

%sociological interest. 
The micro-canonical ensemble is draconian: the number of particles is prescribed, all other configurations get weight 0. %if don't fit bill get weight 0. 
In the grand canonical ensemble, each configuration contributes. There is a value of $\mu$ where the density is the same; at that value the local average of the draconian system is asymptotically the same at that of the more relaxed system. This $\approx$ becomes $=$ when you take the thermodynamic limit,
\bal
\wt\La & \to \Z^d\\
N & \to \iy\\
\fc{N}{\wt \La} & \to \rh.
\end{align*}

%If all that matters it the number of particles in $\La$, then we care about the induced distribution on $\La$. %What is the probability distribution 
What is the induced distribution of the micro-canonical ensemble on $\La$? Under the micro-canonical ensemble what is the probability that $n_{\La}$ (the restriction to $\La$) 
%n_{\Ga_n}
takes a particular value with $\sum_{n}n_x=k$? %If you specify a configuration in $\L$
We count the number of ways to complete the configuration in $\La^c=\wt\La\bs \La$:
\bal
\fc{|\set{n_{\La^c}}{\sum_{x\in \La^c}n_x=N-k}|}{C}
\end{align*}
where $C$ is a normalization constant.

The number of configurations of $M$ particles in volume $V$ is $\binom{V}{M} = \fc{V!}{M!(V-M)!}$. Using Stirling's approximation
\[
\ln (M!) = M(\ln M-1)(1+o(1)),
\]
%gymnastics of elementary type.
%Boltzmann grade in Vienna
%entropy
letting the \ivocab{entropy} $S$ be the logarithm of the number of configurations, %$S=k\ln W$,
\[
\binom{V}{M} = \fc{V!}{M!(V-M)!} 
=: e^{S(M,V)}
\approx e^{Vs(\rh)}
\]
(do this calculation as an exercise)
where 
\[
s(\rh) = -[\rh \ln \rh + (1-\rh)\ln (1-\rh)].
\]
Shannon also found such a formula for entropy.

This attains maximum at $\ln 2$ at $\rc2$ where it has quadratic behavior.

The implication is that if you slightly change the density, the number of configurations changes drastically. In physical substances $V$ may be $10^{23}$. The change is $e^{10^{23}\De s}$. In any average over configurations, only those at the peak contribute, ``winner takes all."

What is the probability of observing $k$ particles in the small box given $n$ in the big box?
\bal
\Pj(n_{\La}) &\approx \fc{e^{|\La^c|s\pf{N-k}{|\wt \La| - |\La|}}}{C}\\ %\quad \rh=\fc{N}{|\wt{\La}|}
%volume of completent times density.
\fc{N-k}{|\wt\La|-|\La|}&=\rh - \fc{k}{|\wt\La|-|\La|}\\
s\pf{N-k}{|\wt\La|-|\La|} & \approx s(\rh) - s'(\rh)\fc{k}{|\La^c|}
%particles here is small fraction of number of outside.
\end{align*}
Changing $k$ by a little bit affects how many particles are outside but not so much the density outside: the correction term is small. Hence for $\sum_{x\in \La}n_x$,
\[
\Pj(n_\La) = \fc{e^{|\La^c|s(\rh)}e^{s'(\rh)}k}{C}.
\]
$e^{|\La^c|s(\rh)}$ is a huge factor but it does not vary with $k$ so we can omit it. This is then ($\mu = e^{-s'(\rh)}$)
\[
=\fc{e^{-\mu k}}{\sum_{n'\in \Om_{\La}}e^{-\mu\sum_{x\in \La} n'_x}}.
\]
The rest of the system acts on the small system as a ``particle (heat) bath."

Here we used very explicit machinery.

 Note we can also apply this method where there are more energy constraints. Make a list of energy constraints; there is a generalization of the equivalence where we averaging over configurations where the constraints have prescribed values. %meet the constraint. All of them get equal weight.
Functions which depend on a small region, can be computed with Gibbs factors, $e^{-\mu N(n) - \be\cal E(n)}$ where $N(n)$ is the number of particles, and $-\mu N(n)$ is the Gibbs factor, and $\cal E(n)$ the energy.

How can we construct an alternative method without the Stirling formula?

Define
\[
Z_{\wt \La} = \sum_{n\in \Om} e^{-\mu N(n)}.
\]
This can be easily computed without Stirling. From the value of this, you can learn the value of the entropy function:
\bal
&=\sum_{n\in \Om} \prod_{x\in \wt\La} e^{-\mu \one[n_x=1]}\\
&=\prod{x\in \wt{\La}}(1+e^{-\mu}) = (1+e^{-\mu})^{|\wt \La|}.
\end{align*}
However, 
\bal
Z_{\wt \La} &=\sum_{n\in \Om} e^{-\mu N(n)}\\
&= \sum_{K\in \N} e^{-\mu K}e^{V s\pf{K}{v}}\\
&= \sum_{K\in \N} e^{V(s\pf{K}{V}-\mu \fc{K}V)}.
\end{align*}
For each $k$ count how many configurations have that value of $k$. Here $S(V,K)=Vs\pf{K}{V}$.
Suppose we find it acceptable to say the system \emph{has} an entropy.

The maximal value of $\fc KV$ takes it all: this is
\[
=e^{V\max_{\rh\in [0,1]}[s(\rh)-\mu\rh]}.
\]
This is the \ivocab{Legendre transform} of the entropy.
Let
\beq{eq:s-leg}
s^*(\mu):= \max_\rh [s(\rh) - \mu\rh] = \ln (1+e^{-\mu})
\eeq
Next time we'll discuss how to derive from this expression the formula for $s(\rh)$, using the inverse Legendre transform.

``The elementary problems are the most precious, once you absorb them they are part of your makeup."