
\blu{4-14-16}

Remarks on DLR condition: For any measure $\mu$ on $(\Om, \cal B)$, we have a map
\[
P_{\La^c}(f) = \E(f|\si_{\La^c}) = \E_{\cal B_{\La^c}}(f|\si),
\]
%there is no Santa Claus. Very large system.
giving the expected value of a function which depends on everything, conditioned on the spins outside $\La$.

The towering property is the following: For $\La_1\sub \La_2\sub\cdots$, for all $m>n$,
\[
(P_{\La_m^c}P_{\La_n^c})f = P_{\La_m^c}f.
\]
The measures $\cal B_{\La_n^c}$ are decreasing.

%two step produces same as get if just freeze and 
%$\si$-algebra depends on local functions. Build all functions based on local functions.
%functions with bizarre property
$\cal B_0$ are ``local functions": each only depends on variables in a finite region. An infinite sum of these may not be local, for example, $\sum_{|x|\in \Z^d} e^{-\al|x|} \si_x\nin \cal B_0$. However, up to $\ep>0$ it can be approximated uniformly by local functions.

$\cal B$ is the smallest collection of functions containing the local functions and that is closed under the usual operations.
%can take $\E$ of nonlocal but in here.

For all finite $\La$, $P_{\La^c}$ is an orthonormal projection onto $\set{f\in L^2(d\mu)}{f\in \cal B_{\La^c}}$. 
%strong convergence
%proven using martingales.
%Family of projections
\begin{thm}[Based on martingale convergence theorem]
For all bounded $f$, 
\[
\E_{\cal B_\iy} (f|\si)=\lim_{n\nearrow \iy} P_{\La_n^c} f(\si) 
\]
exists for $\mu$-almost every $\si$.
%only depends on asymptotic behavior of configuration.
\end{thm}
The averages stay the same but have diminishing fluctuations.

There are $\si$ for which the limit does not converge, e.g., spins organized in rings of $+$ and $-$.

Gibbs measures are characterized by the fact that the conditional expectations are given by nothing else other than the DLR conditions. 
%$\mu$-independent standard formula.

%Fixing structure of...
%for many potentially different...
%For example, another configuration that is the same as $\cdots +++\cdots $ at $\iy$ is any configuration that differs from it by a finite number of flips.
Any two configurations that differ by a finite number of spins are the same at $\iy$. 

%We care about $\varlimsup_{\al\searrow 0} \al \sum e^{-\al|x|}\si_x$.
%parametrize Gibbs states

%how compute if bounded but not local?

Suppose you have a function which is measurable and bounded. Suppose we only know integrals of continuous functions. We can take pointwise limits. Use Monotone Convergence Theorems.

Given a measure on the unit interval, how do we read the integral of measurable functions which are not continuous?

%prob of inner contour exp small.

Remarks on the Mermin-Wagner Theorem: In 2 dimensions there is no symmetry breaking. 
%rotated gradually. 
The idea was to look at the energy penalty for gradual rotation. 
To estimate $H(\wh R_\te\si) - H(\si)$, we bounded the second-order term $\de_2(H)\le c\sum_{\{x,y\}, |x-y|} |\te_x-\te_y|^2$.

Using discrete notation, we saw that (for $r=L, R=2L$)
\[
\min_{\te_x = \begin{cases}
0,&|x|=R\\
1,&|x|=r.
\end{cases}} \sum_x|\nb \te|^2\le cL^{d-2}
\]
We could do better:
\[
\min
\set{\int_{r<|x|<R} |\nb \te|^2 \,d^2x }{\te(x) = \begin{cases}
0, & |x|=R\\
1, & |x|=r.
\end{cases}
}.
\]
The solution is a harmonic function, satisfying $\De \te=0$:
\[
\te(x) = \fc{\ln \prc{R}{|x|}}{\ln \fc{R}{r}},\quad \ln |x| = \Re \ln x.
\]
We have $\nb \te =\rc{\ln \fc Rr} \rc{|x|}$, and
\[
\int_{r<|x|<R} |\nb \te|^2 d^2x = \fc{1}{\ln \pf{R}{r}}.
\]
%multiply by divergent factor. 
In the energy estimate, the second-order term can be made as small as desired.

\section{Random field models}

Let 
\[
H_0=-\sum J_{x-y} (\ul \si_x \cdot \ul \si_y) = \sum_A J_A \Phi(\si_A);
\]
this is rotationally invariant.

Consider now
\[
H = H_0 - \sum (\ul h + \ep \ul\eta_x , \ul \si_x);
\]
we couple with an external field $\ul h$ that encourages the spins to line up with it. However, there is disorder; $\ul\eta_x$ are iid with $|\eta_x|=1$. What are the statistical mechanics?
%spins line up with field
%selected at random but with measure selected at uniform.

Would the system remember boundary conditions or would the local disorder $\ul \eta_x$ dominate?

\begin{thm}[Imry-Ma (Aizenmann-Wehr)]
For such systems, in $D\le 4$, for such a system with continuous symmetry, 
\[
\psi(\be, \ep, \ul h) 
\]
is differentiable at $\ul h=0$.
\end{thm}
Why is the pressure of such a system defined in the infinite-volume limit? We first generate a random environment. 
%convex in $\be$.
It is defined because the way we prove existence of thermodynamic limit was to partition it into noncommunicating regions. If you add randomness, %at that stage of the argument, 
the law of large numbers for independent variables implies that the $\psi(\be, \ep, \ul h)$ is independent of $\eta$ almost surely (independent of disorder). 

The pressure and free energy exist. With frozen disorder, 
%without, want to collaborate.
If there is initally some bias, then we maintain it. The disorder itself may create some symmetry breaking. In low dimensions the disorder plays a decisive role.

\section{Proof of symmetry-breaking of continuous symmetries
}

Let's return to homogeneous systems. We are coming to research-level questions. There is a dearth of arguments to prove symmetry breaking for continuous spaces. 

\subsection{The spin-wave perspective}
%in this room

The key results were by Fr\"ohlich-Simon-Spencer, Fr\"ohlich-Israel-Lieb-Simon.

We can write 
\[
H(\si) = \rc 4 \sum_{x,y} |\ul \si_x - \ul \si_y|^2.
\]
(Note that $|\ul\si_x|^2 + |\ul \si_y|^2$ is constant, while the cross-term is what we had originally.)
Define
\[
\wh \si (p) = \rc{\sqrt{|\La|}} \sum_{x\in \La} e^{i\ul p \cdot \ul x} \ul \si_x.
\]
Being invariant under translations, we can express $H$ as a sum of Fourier modes,
\[
H(\si) = \rc 4\sum_{\ul k} \cal E(k) |\wh \si(k)|^2
\]
where $\ul k \in \fc{2\pi}{L}\Z^d \cap [-\pi, \pi]^d$,
the momenta for which the function is periodic.

You may be tempted to say the problem is now trivial. There is energy for each mode and no coupling. Had this been the end of the story, we have a bunch of independent random variables. However, they are not independent. There are bizarre correlations between the Fourier modes. If we ignored them we would have a quadratic interaction and the equipartition law says that the expected energy per mode is on the order of 
\[
\cal E(k) |\wh \si(k)|^2 \approx \rc{2\be}.
\]
Under certain conditions (reflection positivity), we can prove
\[
\cal E(k) |\wh \si(k)|^2 \le \rc{2\be}.
\]
We'll show how to derive this and how to conclude the theorem from it. It's a beautiful argument but it's a crystal vase. When you have reflection positivity, you use this physically motivated argument; when you don't, the vase breaks.













