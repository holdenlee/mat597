
{\color{blue}3/24/16}

Introduce $t=\frac{\beta-\beta_c}{\beta_c}$. 
What momenta contribute to the rise in the specific heat? That which $E$ is on the order of $Y+Y^{-1}-2$. Do some dimensional analysis. 

The specific heat is 
\be
C=\frac{\partial}{\partial T} \left\langle {H}\right\rangle_\beta = \beta^2\frac{\partial^2 }{\partial {\beta}^2}\Psi = \cdots \ln|\beta-\beta_c|+\cdots 
\ee

Certain fermionic structures pop up. We start from a classical system and out of that the excitations take the form of quantum objects, like quantum spinors (B Kaufman and Lee Scholz). This employed the 2D transfer matrix. When you try to solve it, you transfer from 1 layer to another; the dimension of the matrix is linearly proportional to $n$. In 2D, for reasons to do with integrable systems in 1D, the transfer matrix can be diagonalized. 


The loops we generate can have 1 edge multiple times. We have to disentangle the system of graphs which correspond to configurtions of loops. 

In a high dimension, loops typically miss each other. Perhaps Onsager's solutions may emerge in higher dimensions. (Dimension 3 is the hardest.)

Weight each loop by the product we want. 
%periodic orbit, multiply.



%The world is really quantum
