
\blu{4/12}

%I.e. for every local $f\in \cal B_0$, $\mu(R_\al f)=\mu(f)$ for all Gibbs states $\mu$.

It suffices to prove in 2D there exists $D$ such that for all Gibbs states $\mu$,
\[
\mu(R_\al f) \le D\mu(f)
\]
for all positive $f\in \cal B_0$.
%rotation invariance of a priori measure
%still get a Gibbs measure.
Note that the measure $\wt \mu(f) = \mu(R_\al f)$ is also a Gibbs state. 

To see this, consider a finite volume Gibbs state
\[
\mu_{\La}(f) = \int f(\si_\La) \fc{e^{-\be H^{\text{b.c.}}_\La(\si_\La)}}{Z_{\La}^{\text{b.c.}}} \rh_0(d\si_\La).
\]
Applying a rotation to this gives
\begin{align}
\wt\mu(f) := \mu(R_\al f)&= \int f(\ub{R_\al\si_\La}{\eta}) \fc{e^{-\be H^{\text{b.c.}}_\La(\si_\La)} }{Z_{\La}^{\text{b.c.}}} \rh_0(d\si_\La)\\
&=\int f(\eta) \fc{e^{-\be H^{R_\al^{-1}(\text{b.c.})}_\La (\cancel{R_{\al}^{-1}}\eta)} }{Z_{\La}^{\text{b.c.}}} \rh_0(d\cancel{R_{\al}^{-1}}\eta)\\
&= \mu^{R_{\al}\pat{b.c.}}(f).
\end{align}
(The boundary conditions are rotated by $-\al$.)
The DLR condition is invariant under rotations (exercise).
%identical or mutually singular.

\fixme{See lemma 11.5 in notes. Change $D$ to $C$ and $<$ to $\ge$ in Lemma 11.5.}

%what distinguishes is not local function, but function measurable at $\iy$.
%Choosing a box large enough, you seldom see a difference in magnetization.

%\begin{lem}
%For any system on $\Z^D$ with $O(2)$ symmetries of finite range interactions set
%\[
%\ab{
%\pdxy{}{}
%}
%\]
%\end{lem}

\begin{proof}
Let 
\[
\te(x) = \pi \begin{cases}
1, &\ve{x}\le L_0\\
\fc{2L-\ve{x}_{\iy}}L, & L_0\le \ve{x}\le 2L_0\\
0, &\ve{x}_2\ge 2L_0.
\end{cases}•
\]
(We rotate the configuration in such a way so that the rotation does not extend to $\iy$.)

%\hat R not a uniform rotation, but depends on position.
We will consider the rotation $\wh R_{\fc{\al}{\pi}\te}$ (we use a hat to emphasize that the rotation depends on $x$). 
In the box $\ve{x}\le L_0$, $\wh R_{\fc{\al}{\pi}\te}$ acts as uniform rotation:
\begin{align}
\mu(R_\al f) &= \mu(f(\wh R_{\fc{\al}{\pi}\te}\si))\\
&=\int \ba{
\int_{\Om_{\La}(2L_0)} 
f(\ub{\wh R_{\fc{\al}{\pi}\te} \si_{\wt \La}}{\si_{\wt \La}'}) \fc{e^{-\be H_{\wt \La}(\si_{\wt \La}|\si_{\wt \La^c})}}{Z_{\wt\La}(\si_{\wt \La^c})} \rh_0(d\si_{\wt \La})
} \rh(d\si_{\wt\La^c})\\
&=\int \ba{
\int_{\Om_{\La}(2L_0)} 
f(\si_{\wt \La}') \fc{e^{-\be H_{\wt \La}(\wh R_{\fc{\al}{\pi}}\si_{\wt \La}'|\si_{\wt \La^c})}}{Z_{\wt\La}(\si_{\wt \La^c})} \rh_0(d\si_{\wt \La}')
} \rh(d\si_{\wt\La^c})
\end{align}
When you carry out a gradual rotation, the energy does not change by a lot.
We have
\begin{align}
\mu(R_\al f) &= \mu(\wh R_{\fc{\al}{\pi}\te}f) \\
&= \mu(fe^{-\be [H(\wh R_{\fc{\al}{\pi}\te}\si) - H(\si)]}).
\end{align}
%If we allow self to proceed by leaps of imagination, can arrive here immediately.
Formally, we rewrote
%infinite, go through finite value truncation.
%taper to 0 in a finite distance
\[
f(0) e^{-\be H(\wh R^{-1}\si)} \fc{\rh_0(d\si)}{Z}
= e^{-\be[R^{-1}H - H]} e^{-\be H(\si)}\fc{\rh_0(d\si)}{Z}
\]
Note $R^{-1}H-H$ is finite; what is left is to estimte it. We estimate it in the case $H=-\sum_{|x-y|=1}\ul{\si}_x\cdot \ul{\si}_y$. We calculate
\begin{align}
H(R^{-1}\si) - H(\si) 
&= - \sum_{\text{nearest neighbor}} \ba{
(R_{\te(x)}^{-1}\ul{\si}_x)\cdot 
(R_{\te(y)}^{-1}\ul{\si}_y) - \ul{\si}_x\cdot \ul{\si}_y
}\\
&= \sum_{\text{nearest neighbor}} [\ul{\si}_x \cdot \ul{\si}_y - \ul{\si}_x \cdot R_{\te(x) - \te(y)} \ul{\si}_y]\\
&= \sum_{\text{nearest neighbor}} [\al(\ul{\si}_x\cdot \cal L \ul{\si}_y)]\\
&=\sum_{\text{nearest neighbor}} [\fc{\al}{\pi} \de_1(H) + \de_2(H)],
%taylor expansion 
\end{align}
where $\de_i$ is the $i$th order correction, from the Tylor expansion. Then 
\begin{align}
\mu(R_\al f) &= \mu(\wh R_{\fc{\al}{\pi}\te}f)\\
&= \mu(f e^{-\be [H(\wh R^{-1}_{\fc\al\pi\te}\si) - H(\si)]}\\
&= \mu(f e^{[\fc{\al}{\pi} \de_1(H) + \de_2(H)]})
\end{align}
%boundary increases
%softly rotate. drop $L^{D-1}$ to $L^{D-2}$
%If dimension $\le 2$ the bounded.
with $|\de_2(H)|\le C\sum |\te(x)-\te(y)|^2 \le CL^{D-2}$. 

If the dimension is $D\le 2$ this is bounded. The soft rotation dropped this quantity from $L^{D-1}$ to $L^{D-2}$.

However, $\de_1(H)$ might be bigger; a priori we only have $|\de_1(H)|\le C\sum|\te(x)-\te(y)|=O(L^{D-1})$.
The idea is that we can rotate ``from the other side." To deal with the first order term,
\begin{align}
%\mu(f) &= 
\mu(f) &= \mu(f e^{-\fc{(2\pi-\al)}{\pi}\de_1(H) + \wt \de_2(H)})
\end{align}
The second-order term is controllable, the first-order term is hard to control. We have two expressions where the coefficient of the first-order terms has different signs. Let convexity work for us.

Consider 
\[
Q(t) = \mu(f e^{(1-t)[\fc\al\pi\de_1(H)+\de_2(H)]} e^{t[-\fc{(2\pi - \al)}{\pi}\de_1(H) + \wt\de_2(H)]}).
\]
The function $\ln Q(t)$ is convex in $t$ with $Q(0)=Q(1) = \mu(R_\al f)$. Thus by convexity, or H\"older's inequality,
%convexity or Holder
for all $0\le t\le 1$, 
\[
Q(t) \le Q(0)^{1-t}Q(1)^{t} = \mu(Rf).
\]
%Both terms are uniformly bounded
$(1-t)\al - t(2\pi - \al) = 0$. At $t=t_\al$,
\[
Q(t) = \mu(f) e^{(1-t) \de_2+ t\wt{\de}_2} \ge e^{-CL^{D-2}} \mu(f)
\]
giving us $\mu(R_\al f) \ge e^{-CL^{D-2}}\mu(f)$.

%Expected vlue of rotatied is identical to original times difference produced 
%look at cost of rotating spins gently.
\end{proof}
What gave physicists the intution to look at second order term is that at low temperature, when the spins line up, the first order term vanishes. The second-order term is $\fc{L^d}{L^2}$. But there is chaos and term-by-term it's not true that the first-order term can be ignored. We sidestep this in the analysis.
%If you ask how the sysem responds to gentle rotation, the first derivative should not play a role.

If $D>2$, the bound says nothing.


The theorem implies that the pressure $\Psi(\be, \ul h)$ is differentiable in $\ul h$ at $\ul h = \ul 0$. We know it's symmetric, but it could have a kink or be smooth with zero derivative. The theorem tells us it is smooth.

The argument works equally well for quantum systems and for systems with disorder. For a while people had two predictions based on 2 theories. Disordered systems in $D$ dimensions behaved as ordered systems in $D-2$ dimensions. Imry and Ma predicted that for $D\le 4$ there is no symmetry breaking. it uses this technology with bootstrapping.
%instead that continuous symmetry breaking is pushed 

Rigorous proofs arrived before physicists had a consensus.
%\begin{thm}[Imry-Ma Theorem]
%
%\end{thm}

Symmetry breaking occurs in dimensions $D\ge 3$. %reflection positivity.

